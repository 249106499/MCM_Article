\documentclass[UTF8]{article}
\usepackage[utf8]{inputenc}

\title{当我们谈论社会时,我们在谈论什么}
\author{nobody}
\date{\today}

\usepackage{amsmath}
\usepackage{amsfonts}
\usepackage{amssymb}
\usepackage{graphicx}
\usepackage{ctex}
\usepackage{multicol}
\allowdisplaybreaks
\bibliographystyle{ieeetr}
\begin{document}
\maketitle
\begin{abstract}
接下来会涉及对\emph{消费,景观}的社会学分析。

真实是符号抵达不了的地方,真实是褪去了符号后,剩下的。
\end{abstract}
\section{消费社会}
凯恩斯,经济问题并不是人类的永恒问题。

物资匮乏的问题得到解决,即\emph{丰盛}的来临。

《资本论》对资本主义的分析无疑是深刻的,但是资本主义还是存在着,
\subsection{符号消费}
物品是消费文化中可见的部分。

符号消费,售价脱离使用价值,更多的取决于该品牌在符号体系中的位置

如果将需求看作对社会意义的需求,那么他就是变动不定且永远无法得到满足的

时尚,代际商品按次序,有预谋的迅速退场,商品在过时之后马上又有新的商品来取代它,人们(消费者)获得的物品却无比平庸

\subsection{文化工业}
法兰克福学派学者认为文化的基本功能是“否定”和“对幸福的承诺”。

“我们不对读者负责”

time killer

“读者就是上帝”

所有适应了新环境文化的人(而且在此范围内,即使是有教养的人也不例外,或者说不会例外),并没有权利参与到文化中去,他们有权参加的是文化再循环。

文化的商品化,文化只有成为商品并进入市场,才能得到关注

快速,时髦,肤浅,片断化,理论成为一种“超级商品”,成为无思时代兜售和宣扬最时髦消费意识和人生态度的一种谎言工具。

媚俗,对符号的挪用,拼贴一种独特的价值贫乏,失去原作精神的滑稽模仿,“接地气”

\section{景观}
意象交织成的网络

景观的本质是拒斥对话,媒介即信息

人只能如此看,如此听,如此想,人成为媒体的终端接收器
\subsection{广告}
“广告不让人去理解,也不让人去学习,而是让人去希望,在此意义上,它是一种预言性话语。”

广告是自我实现的,自我重复的且逻辑循环的寓言,它告诉我们应该如何通过消费获得幸福

\subsection{拟象}
没有本体的模仿

攻壳的英文名Stand alone complex
\subsection{大众传媒}
当代价值的命名者

新闻,超真实
\section{我有一个朋友,他三观很正}
他认为这一切都是理所当然。完全没有必要为此大惊小怪。

享受着消费社会的便利,还要批判它,不太厚道吧。
\section{你能一辈子不看电影吗}
不能
\section{后话}
广告超越了真和伪,时尚超越了美和丑,当代物品超越了有用无用

Raymond Carver
%\bibliography{tex}
\end{document}
